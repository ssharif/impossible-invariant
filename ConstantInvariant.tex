\documentclass{amsart}
\usepackage{amsmath, amsthm, amssymb,latexsym,enumerate,mathrsfs}
\usepackage[all]{xy}
\usepackage{enumitem}
\usepackage{hyperref}
\usepackage{invariant}
\usepackage{xcolor}
\usepackage{tikz-cd}
\usepackage{verbatim}
\usepackage{lineno}
\linenumbers

\title[Algebraic maps constant on unpolarized isomorphism classes]{Algebraic maps constant on isomorphism classes of unpolarized abelian varieties are constant}

\author[E.\ Rains]{E.\ Rains}
\address{}
\email{}
\author[K.\ Rubin]{K.\ Rubin}
\address{}
\email{}
\author[T.\ Scholl]{T.\ Scholl}
\address{}
\email{}
\author[S.\ Sharif]{S.\ Sharif}
\address{}
\email{}
\author[A.\ Silverberg]{A.\ Silverberg}
\address{Department of Mathematics, University of California, Irvine, CA 92697, USA}
\email{asilverb@uci.edu}
%\subjclass[2010]{??}
\keywords{abelian varieties, polarizations}
\thanks{Support for the research was provided by the Alfred P.~Sloan Foundation
and the National Science Foundation.}

\begin{document}

\begin{abstract}
?
\end{abstract}


\today
\maketitle

%\tableofcontents


\section{Introduction}
\label{sec:introduction}

If $A$ and $B$ are abelian varieties, we say $A$ and $B$ are \emph{weakly isomorphic}, written $A \approx B$, if $A$ and $B$ are isomorphic as unpolarized abelian varieties. Let $\ag$ denote the moduli space of principally polarized abelian varieties of dimension $g$. Our main result is the following.
\begin{theorem}\label{thm:invariant-c-constant}
  Suppose that $R$ is a subring of $\C$ and $f: \ag \to X$ is a morphism of $R$-schemes. Suppose $f(A) = f(B)$ whenever $A \approx B$. Then $f$ is a constant function.
\end{theorem}

% Let $R$ be a scheme, and for an $R$-scheme $S$ write $S[R]$\textcolor{red}{TODO: is this necessary?} to be the set
% \[
%   \dlim_{T/R} S(T)
% \]
% where the limit is over all $R$-schemes $T$. Let
% \[
%   \sg: \rschemes \to \mathcal{P}((\ag \times \ag)[R])
% \]
% be the functor from $R$-schemes to the power set of $(\ag \times \ag)[R]$ that sends an $R$-scheme $T$ to the set
% \[
%   \sg(T) = \{(A,B) \in (\ag \times \ag)(T) | A \approx B\}.
% \]
If $R$ is a scheme, let
$$
\sg(R) = \{(A,B) \in (\ag \times \ag)(R) | A \approx B\}.
$$
We will deduce Theorem \ref{thm:invariant-c-constant} from the following result.

\begin{theorem}\label{thm:sg-c-dense}
  If $R$ is a subring of $\C$, then the set $\sg(R)$ is Zariski dense in the $R$-scheme $\ag \times \ag$.
\end{theorem}
Since a scheme over $\C$ is also scheme over $R$, it suffices to prove Theorems \ref{thm:invariant-c-constant} and \ref{thm:sg-c-dense} when $R = \C$.

In arbitrary characteristic, we have the following result, which will be proved in Section~\ref{sec:characteristic-p}.
\begin{theorem}\label{thm:arbit-char}
  Suppose that $R$ is a scheme and $f: \ao^g \to X$ is a morphism of $R$-schemes. Suppose $f(A) = f(B)$ whenever $A \approx B$. Then $f$ is a constant function.
\end{theorem}

The motivation behind this result is the construction of a cryptographic protocol in~\cite{multiparty}. In that protocol, $n$ parties each construct a product of elliptic curves over a finite field such that any pair of products is isomorphic. Or, to put it another way, each party computes a point of $\ao^g$ so that the chosen points are weakly isomorphic to each other. However, the protocol is incomplete. It remains to be able to extract a numerical invariant of the product of elliptic curves which respects weak isomorphism; that is, a map $f: \ao^g \to X$ for a suitable space $X$ such that $f(A) = f(B)$ whenever $A \approx B$. In this context, Theorem~\ref{thm:arbit-char} shows that if $f$ is algebraic, then it is constant, which is not useful cryptographically. There may be useful non-algebraic invariants; indeed, \cite[Section 5]{multiparty} considers an invariant due to Deligne, but unfortunately finds it flawed as well.

%\subsection{Overview}

\begin{definition}\label{def:detl}
If $\ell$ is a prime number and $g$ is a positive integer, let $\detl$ denote the set of positive definite symmetric $g \times g$ integer matrices whose determinant is a power of $\ell$.
\end{definition}

If $N$ is a positive integer, let $Y_0(N)$ denote the modular curve parametrizing pairs $(E, C)$, where $E$ is an elliptic curve and $C \subset E$ is a cyclic subgroup of $E$ of order $N$. For every $A \in \detl$, define a map $\psimod: Y_0(\det(A)) \to \ag$ as follows. If $(E, C) \in Y_0(N)$, the matrix $A$ induces a natural endomorphism $\lambda_A$ of $E^g$. Since $A$ is symmetric and positive definite, we can view $\lambda_A$ as a polarization on $E^g$. Let $B = E^g/((\ker \lambda_A) \cap C^g)$, let $\pi: E^g \to B$ be the quotient map, and let $\lambda$ denote the (unique) principal polarization on $B$ such that $\pi^*(\lambda) = \lambda_A$. Define $\psimod$ by sending $(E,C) \in Y_0(N)$ to the abelian variety $B$ with principal polarization $\lambda$, and let $X_A$ denote the image of $\psimod$.

\begin{theorem}\label{thm:curves-dense}
  The set $\bigcup_{A \in \detl} X_A$ is Zariski dense in $\ag$.
\end{theorem}

\begin{theorem}\label{thm:Sg-dense}
  If $A,A' \in \detl$, then $\sg(\C) \cap (X_A \times X_{A'})$ is Zariski dense in $X_A \times X_{A'}$.
\end{theorem}

Theorems \ref{thm:curves-dense} and \ref{thm:Sg-dense} are proved in sections \ref{sec:step-1} and \ref{sec:step-2}, respectively.

\textcolor{red}{TODO: introduce/move}

\begin{proof}[Proof of Theorem~\ref{thm:sg-c-dense}]
  If $X$ and $Y$ are subsets of $\ag \times \ag$, we write (I suggest omitting this notation and writing it out in words, rather than proving the result in a long expression) $X \prec Y$ if $X \subseteq Y$ and $X$ is Zariski dense in $Y$. Then
  \begin{align*}
    \sg(\C)
    &\supseteq
    \sg(\C) \cap \left(\bigcup_{A,A'} X_A \times X_{A'}\right)
    \\
    &=
    \bigcup_{A,A'} \sg(\C) \cap (X_A \times X_{A'})
    \\
    &\prec
    \bigcup_{A,A'} X_{A} \times X_{A'}
    &&\text{By Theorem~\ref{thm:sg-c-dense}}
    \\
    &=
    \left(\bigcup_{A} X_A\right) \times \left(\bigcup_{A'} X_{A'}\right)
    \\
    &\prec
    \ag \times \ag
    &&\text{By Theorem~\ref{thm:Sg-dense}}.
  \end{align*}
\end{proof}

\textcolor{red}{TODO: introduce/move}

\begin{proof}[Proof of Theorem~\ref{thm:invariant-c-constant}]
  Suppose $f: \ag \to X$ is a morphism of $R$-schemes. Consider the diagonal $\df$ determined by the diagram
  \[
  \begin{tikzcd}
    \df \arrow[r] \arrow[d] & \ag \arrow[d,"f"]
    \\
    \ag \arrow[r,"f"] & X.
  \end{tikzcd}
  \]
  There is a natural map $\df \to \ag \times \ag$. The hypotheses on $f$ will imply that, viewed as functors, the latter map factors through $\sg$.
  Assume $f$ is separated \textcolor{red}{TODO: is this necessary?}.
  % https://ncatlab.org/nlab/show/separated+morphism+of+schemes
  Then $\df$ is closed in $\ag \times \ag$. By Theorem~\ref{thm:sg-c-dense}, this implies that $\df = \ag \times \ag$, whence the claim follows.\textcolor{red}{TODO: whence? I think this needs more.}
\end{proof}



\section{The curves $X_{A}$}
\label{sec:curves-on-Ag}

(This section should be shortened, if possible.)

Fix a positive integer $g$, and a $g \times g$ symmetric, positive definite integer matrix $A$ whose determinant $N$ is not divisible by the characteristic of our base. %(Later we will take $N$ to be a power of a prime $\ell$.)
Recall the map
\[
\psimod: Y_0(N) \to \ag, \quad (E,C) \mapsto (B,\lambda)
\]
defined in the introduction.


%\textcolor{red}{TODO: what constraints are required for $n$?}. In this section we will construct the curves $X_A \subset \ag$. % mentioned in Section~\ref{sec:introduction}.

%We will define the maps $Y_0(\ell^n) \to \ag$ in two ways: geometrically based on the moduli interpretation, and complex analytically. We will use that these two interpretations are the same in the proof of Theorem~\ref{thm:sg-c-dense}. We will apply the moduli interpretation to prove Theorem~\ref{thm:Sg-dense}, and the complex analytic version to prove Theorem~\ref{thm:curves-dense}. Since the latter result does not directly carry over to positive characteristic, we will use a variant that shows density in $\ao^g$; see \ref{sec:characteristic-p}.

For another construction of these maps, see \cite[p. 19 et seq.]{rains}.

\begin{definition}
If $M \in \Gl_g(\Z)$ and $E$ is an elliptic curve, let $\rho_M$ denote the natural endomorphism of $E^g$ induced by $M$.
\end{definition}

\subsection{Geometric description}
\label{sec:geo-desc}


%Suppose $(E, C) \in Y_0(N)$, i.e., $E$ is an elliptic curve and $C \subset E$ is a cyclic subgroup of $E$ of order $N$.  The matrix $A$ induces a natural endomorphism $\lambda_A$ of $E^g$. Since $A$ is symmetric and positive definite, by identifying $E^g$ with its dual we can view $\lambda_A$ as a polarization on $E^g$ of degree $N^2$.
If $(E, C) \in Y_0(N)$, let
$$
K(A) = (\ker \lambda_A) \cap C^g  \subset E^g.
$$

%Where is the next result used? If we use it, we should refer to it. If we don't use it, let's comment it out, for now.
%\begin{lemma}
%  The subgroup $K(A)$ is a maximal isotropic subgroup of $\ker \lambda_A$.
%\end{lemma}
%
%\begin{proof}
%  Let $P$ be a generator of $C$, defined over the algebraic closure of our base field. Identify $C^g$ with $(\Z/N)^g$ via the map $(a_iP) \mapsto (a_i)$. Then multiplication by the matrix $A$ induces a map $\lambda_A^{(C)}:C^g \to C^g$. One sees (how??) that $\#(\ker \lambda_A^{(C)}) = \sqrt{\#(\ker \lambda_A})$. Further, $\ker \lambda_A^{(C)} \subset C^g$ is isotropic (why?). The claim follows.
%\end{proof}

%Let $B = E^g/K(A)$, and let $\pi: E^g \to B$ be the quotient map. Then by \cite[Prop. 16.8]{milne-av}, there is a principal polarization $\lambda$ on $B$ for which $\pi^*(\lambda) = \lambda_A$. If $X$ is an abelian variety, let $NS(X)$ denote its N\'eron-Severi group. Since $\pi^*: NS(B) \to NS(E^g)$ is injective, $\lambda$ is unique.

%\begin{definition}\label{def:psimod-def}
%Define
%\[
%\psimod: Y_0(N) \to \ag, \quad (E,C) \mapsto (B,\lambda).
%\]
%\end{definition}

(State here where we'll use the next result.)
\begin{lemma}\label{lem:psimod-weakly-isomorphic-to-product}
  Let $d_1, \cdots, d_g$ be the elementary divisors of $A$, and for $(E,C) \in Y_0(N)$ and $1 \le i \le g$, let $E_i = E/d_i^{-1}NC$. If $(B,\lambda) = \psimod(E,C)$, then
$
    B \approx E_1 \times \cdots \times E_g.
$
\end{lemma}

\begin{proof}
  Let $D$ be the diagonal matrix with $d_1, \cdots, d_g$ on the diagonal.
%   \[
%     D =
%    \begin{bmatrix}
%      {d_1} & & & \\
%      & {d_2} & & \\
%      & & \ddots & \\
%      & & & {d_{g}}
%    \end{bmatrix}
%  \]
Then $D$ is the Smith normal form of $A$, and there exist $U,V \in \Gl_g(\Z)$ such that $A = UDV$. Then $\rho_U$ and $\rho_V$ are automorphisms of $E^g$ induced by $U$ and $V$, respectively, and $\rho_A = \rho_U \rho_D \rho_V$. Since
  \begin{align*}
    \ker \rho_D &= E[d_1] \times E[d_2] \times \cdots \times E[d_g]
  \end{align*}
  we have $(\ker \rho_D) \cap C^g = \prod_{i=1}^g d_i^{-1}NC$. %d_1^{-1}NC \times \cdots \times d_g^{-1}NC$.
 Since $\rho_V(C^g) = C^g$, we now have
  \[
    \rho_V((\ker \rho_A) \cap C^g) = (\ker \rho_D) \cap C^g.
  \]
  Thus the quotients of $E^g$ by $(\ker \rho_A) \cap C^g$ and by $(\ker \rho_D) \cap C^g$ are weakly isomorphic. The lemma follows.
\end{proof}

\subsection{Analytic description}
\label{sec:ana-desc}

In this section we assume that the base scheme is $\C$.
%Suppose that $A$ is a symmetric, positive definite integer matrix. Let $N = \det A$.
Let $\hh$ denote the complex upper half plane and let $\hh_g$ denote the degree $g$ Siegel upper half space. Recall that $\Sp_{2g}(\Z)$ (resp., $\Gamma_0(N)$) acts on $\hh_g$ (resp., $\hh$) by fractional linear transformations,
 $Y_0(N) = \hh/\Gamma_0(N)$, and $\ag = \hh_g/\Sp_{2g}(\Z)$.

\begin{lemma}\label{lem:psi-A-n}
The map \[
  \psi_A: \hh \to \hh_g, \quad  \tau \mapsto \tau A
\]
induces a map %descends to a morphism
$
  \psimodt: Y_0(N) \longrightarrow \ag.
$
\end{lemma}

\begin{proof}
Let
\[
\sigma = \begin{bmatrix} a & b \\ c & d \end{bmatrix} \in \Gamma_0(N) \text{ and } M = \begin{bmatrix} aI_g & bA \\ cA^{-1} & dI_g \end{bmatrix}
\]
where $I_g$ denotes the $g \times g$ identity matrix.
If $\tau \in \hh$, a direct computation shows that $M \in\Sp_{2g}(\Z)$
% it is integral because \det(A) | c, so cA^{-1} is integral.
% M is in Sp because then M^t*\Omega*M = M
% where \Omega = \begin{matrix} 0 & I \\ -I & 0 \end{matrix}
and  $\sigma(\tau)A = M(\tau A)$.
%(It's well-defined. Why is it a morphism?)
\end{proof}


%In fact, $\psimodt$ is none other than $\psimod$, from which one can obtain an alternative proof of Lemma \ref{lem:psi-A-n}.
\begin{lemma}
  When the base scheme is $\C$, we have $\psimodt = \psimod$.
\end{lemma}

\begin{proof}
  Suppose $(E, C) \in Y_0(N)$. There exists $\tau \in \hh$ such that $E = \C/(\Z + \tau \Z)$ and $C = \langle 1/N \rangle$. We first show that $\psimodt(E,C) \approx \psimod(E,C)$.

  Let $\Lambda = \Z^g + \tau \Z^g \subset \C^g$. Then $\C^g/\Lambda = E^g$, where projection on the $i$th complex coordinate is the same as projection onto the $i$th factor of $E^g$. Let
  \[
    \tla = \Z^g + \tau A \Z^g \quad \text{ and } \quad     \Lambda_A = A^{-1}\Z^g + \tau \Z^g.
  \]
By definition, $\psimodt(\tau) \approx \C^g/\tla$.
%  \[
%    \Lambda_A = A^{-1}\Z^g + \tau \Z^g.
%  \]
%  Multiplication by $A^{-1}: \C^g \to \C^g$ induces an isomorphism
%  \[
 %   \C^g/\tla \xrightarrow{\sim} \C^g/\Lambda_A.
 % \]
 Since $\Lambda \subset \Lambda_A$, there is a natural isogeny
  \[
    \rho: E^g = \C^g/\Lambda \to \C^g/\Lambda_A.
  \]
  We wish to compute $\ker \rho$.

  Multiplication by $A: \C^g \to \C^g$ induces an isomorphism
  \[
     \C^g/\Lambda_A \xrightarrow{\sim}  \C^g/\tla
  \]
and an isogeny
  \[
    \rho_A: \C^g/\Lambda \longrightarrow \C^g/\Lambda,
  \]
  %This is the same isogeny $\rho_A$ as in the proof of Lemma \ref{lem:psimod-weakly-isomorphic-to-product}; i.e., it
  where $\rho_A$ as before is the natural isogeny $E^g \to E^g$ induced by the matrix $A$. Then
\[
\ker \rho_A = A^{-1}\Z^g + \tau A^{-1} \Z^g \pmod{\Lambda}.
\]
%With our identification $E = \C/(\Z + \tau\Z)$, recall that the distinguished subgroup $C \subset E$ is $\langle 1/N \rangle$. Then $C^g \subset E^g$ is $\frac{1}{N}\Z^g \pmod{\Lambda}$. It follows that
Then
\[
\ker \rho = (\ker \rho_A) \cap C^g = (\ker \lambda_A) \cap C^g = K(A).
\]
Since $\psimod(E,C) \approx E^g/K(A)$, we have proven that $\psimod(E,C) \approx \psimodt(E,C)$.

  A polarization on a $g$-dimensional complex abelian variety is given by an alternating Riemann form $E: \C^g \times \C^g \to \R$.

  For a point $\Omega \in \hh_g$, the corresponding abelian variety is
\[
\C^g/(\Z^g + \Omega \Z^g),
\]
and the corresponding alternating Riemann form is the unique such form $E$ satisfying, for $u,v \in \Z^g$, $E(u,\Omega v) = u^tv$. We call this form the \emph{Siegel alternating form} arising from $\Omega \in \hh_g$; this form specifies the principal polarization for the class of $\Omega$ in $\ag$. %For example, if $\Omega = \tau I$ for $\tau \in \hh$, so that the abelian variety is $E^g$, then the Siegel alternating form represents the product principal polarization.

Now consider the (nonprincipal) polarization $\rho_A$ on $E^g$. The corresponding alternating Riemann form is given by $E_A(u,\tau v) = u^tAv$, where $u,v \in \Z^g$. Since $E_A(\Lambda_A \times \Lambda_A) \subset \Z$, it follows that $E_A$ is a Riemann form for $\C^g/\Lambda_A$ as well. Write $\lambda$ for the polarization of $\C^g/\Lambda_A$ given by $E_A$. Recall that the polarization coming from $\psimod$ is the unique one descending from the product polarization on $E^g$. Since $\lambda$ also descends from the product polarization, we have
\[
  \psimod(E,C) = (\C^g/\Lambda_A, \lambda).
\]
Let $\tilde{\lambda}$ be the polarization coming from the Siegel alternating form for $\tau A$; that is, given by the form $E_{\tau A}(u, \tau A v) = u^tv$. If $u, v \in \Z^g$, we have
\begin{align*}
  E_{\tau A} (u, \tau A v) &= u^tv \\
                          &= E_A(A^{-1}u, \tau v) \\
                          &= E_{A}(A^{-1}u, A^{-1}\tau Av).
\end{align*}
Thus multiplication by $A$ on $\C^g$ induces an isomorphism of polarized abelian varieties
\[
  \psimod(E,C) = (\C^g/\Lambda_A, \lambda) \xrightarrow{\sim} (\C^g/\tla, \tilde{\lambda}) = \psimodt(E,C),
\]
%Since $\psimodt(E,C) = (\C^g/\tla, \tilde{\lambda})$,
as desired.


 % Further, this polarization is specified by letting the columns be a symplectic basis for the associated Riemann form $H: \C^g \times \C^g \to \C$; this is precisely the product polarization.

 %  The abelian variety $B$ associated to $\psi_A(\tau)$ is $\C^g/\tilde{\Lambda}$, where $\tilde{\Lambda}$ is the lattice generated by the columns of
 %  \[
 %    \begin{bmatrix}
 %      I | \tau A
 %    \end{bmatrix},
 %  \]
 %  and with the associated principal polarization.
\end{proof}
% Write $\Omega$ for $\tau I$, where $I$ is the $g \times g$ identity matrix. The period matrix for $\psi_{I,n}(\tau)$ is
% \[
%   M = \begin{bmatrix}
%     I | \Omega
%   \end{bmatrix}.
% \]
% Observe that if $E$ is the elliptic curve corresponding to $\tau \in \hh$, then the abelian variety corresponding to $\psi_{I,n}(\tau) \in \hh_g$ is $E^g$ with the product polarization.

% Given a period matrix $M$, let $\Lambda(M)$ be the lattice generated by the columns of $M$, so that the corresponding abelian variety is $A(M) := \C^g/\Lambda(M)$. We say two period matrices $M, M'$ are \emph{equivalent} if $\C^g/\Lambda(M)$ is isomorphic to $\C^g/\Lambda(M')$, and similarly the matrices are \emph{weakly equivalent} if the abelian varieties are weakly isomorphic. Let $D$ be the Smith normal form of $A$, so that there are $U,V \in \Gl_g(\Z)$ for which $A = UDV$. Then the period matrix for $\Psi_{A,n}(\tau)$ is
% \[
%   \begin{bmatrix}
%     I | \Omega A
%   \end{bmatrix},
% \]
% which is equivalent to
% \[
%   \begin{bmatrix}
%     D^{-1}U^{-1} | V
%   \end{bmatrix}
% \]
% and hence weakly equivalent to
% \[
% M' = \begin{bmatrix}
%     D^{-1}U^{-1} | \Omega
%   \end{bmatrix}.
% \]
% Let $M''$ be the period matrix\[
%   \begin{bmatrix}
%     D^{-1} | \Omega
%   \end{bmatrix}.
% \]
% The matrix $D^{-1}$ is a diagonal matrix whose diagonal entries are powers of $1/\ell$. Suppose the $i$th diagonal entry is $1/\ell^{n_i}$, and set
% \[
%   \vec{n} = (n_1, \ldots, n_g).
% \]
% One sees that $A(M'') = \psivec(\tau)$, where $\lambda$ is the product polarization. In particular, if $P \in E$ generates the distinguished $\ell^n$-torsion subgroup, then
% \[
%   A(M'') = E/\langle \ell^{n-n_1} P \rangle \times \cdots \times E/\langle \ell^{n-n_g} P \rangle.
% \]
% The abelian variety $A(M')$ is similary a quotient of $E^g$ by an $\ell$-power subgroup. But the natural map $U: E^g \to E^g$ sends one kernel subgroup to another, and thus induces a (weak) isomorphism $A(M') \to A(M'')$. As $A(M'') = \psivec(\tau)$ and $A(M')$ is weakly isomorphic to $A(M) = \psimat(\tau)$, we conclude that
% \[
% \psimat(\tau) \textrm{ and } \psivec(\tau)
% \]
% are weakly isomorphic.



\section{Proof of Theorem~\ref{thm:curves-dense}}
\label{sec:step-1}

%The main goal of this section is to prove Theorem~\ref{thm:curves-dense}.

\begin{lemma}\label{lemma:sl-z-1overl-dense-sl-r}
  The set $\Sl_g(\Z[1/\ell])$ is dense in $\Sl_g(\R)$ with respect to the Euclidean topology.
  %Here, the topology is the one induced by the Euclidean metric on the entries.
\end{lemma}

\begin{proof}
  Let $G \in \Sl_g(\R)$. Factor $G$ as a product of elementary matrices
  \[
    G = E_n \cdots E_2 E_1
  \]
  where $\det E_i = \pm 1$. For each $E_i$, the entries, with at most one exception, are $0$ or $\pm 1$. Thus we can find a $g \times g$ matrix $E_i'$ with entries in $\Z[1/\ell]$ and $\det E_i' = \det E_i$ which is arbitrarily close to $E_i$. Let $G' = E_n' \cdots E_1'$. Certainly $G' \in \Sl_g(\Z[1/\ell])$. Since matrix multiplication is continuous, $G'$ is close to $G$.
\end{proof}

% \begin{proof}[Alternative proof]
%   Let $G = [g_{ij}] \in \Sl_g(\R)$. Let $X = [x_{ij}]$ be a matrix of $g^2$ variables, and consider the determinant map $\det: \R[\{x_{ij}\}] \to \R$. Choose $r,s$ so that
%   \[
%     \frac{\partial \det}{\partial x_{rs}}(G) \neq 0.
%   \]
%   Since $\Z[1/\ell]$ is dense in $\R$, for $(i,j) \neq (r,s)$ we can find $a_{ij} \in \Z[1/\ell]$ which is arbitrarily close to $g_{ij}$. By our hypotheses on $(r,s)$, there exists $a_{rs} \in \R$ such that $\det [a_{ij}] = 1$. Clearly $a_{rs} \in \Z[1/\ell]$. The determinant is continuous, so $a_{rs}$ must also be close to $g_{rs}$.
% \end{proof}

% \begin{proof}
%   The group scheme $\Sl_g$ is a connected reductive group, and hence is unirational. (???cite) Since $\Z[1/\ell]$ is topologically dense in $\R$, the claim follows.

%   Details: unirational means there is a dominant rational map $f:\Pro^n \to \Sl_g$ for some $n$. Given $G \in \Sl_g(\R)$, there some $G' \in f(\Pro^n(\R))$ which is close to $G$. Let $P \in \Pro^n(\R)$ with $f(P) = G'$. Then
% \end{proof}

%\begin{definition}\label{def:detl}
%  For $\ell$ a prime, let $\detl$ be the set of $g \times g$ symmetric, positive definite integer matrices $A$ for which %$\det A$ is a power of $\ell$.
%\end{definition}

Recall the definition of $\detl$ in Definition \ref{def:detl}.

\begin{lemma}\label{lemma:ggt-spd-detl}
  If $G \in \Sl_g(\Z[1/\ell])$, then for all sufficiently large $n \in \N$,
  \[
    \ell^n GG^t \in \detl.
  \]
\end{lemma}

\begin{proof}
  If $n \in \N$, the $\ell^n GG^t$ is symmetric and positive definite, and its determinant is a power of $\ell$. Let $N$ be the maximum power of $\ell$ appearing in the denominators of the entries of $GG^t$. Then for $n \geq N$, $\ell^n GG^t$ is an integer matrix.
\end{proof}


\begin{definition}
  Let $\permat$ denote the set of $g \times g$ real, positive definite symmetric matrices  with determinant $1$.
\end{definition}

\begin{lemma}\label{lemma:ggt-periodmatrices}
  $\permat = \{GG^t | G \in \Sl_g(\R)\}$.
\end{lemma}

\begin{proof}
  Suppose $A \in \permat$. By the Spectral Theorem, there exist a real orthogonal matrix $O$ and a real diagonal matrix $D$ such that $A = ODO^t$. Let $d_1, d_2, \dots, d_n$ be the diagonal entries of $D$. Since $A$ is positive definite, $d_i > 0$ for all $i$. Further, $\prod d_i = \det A = 1$. Let
  \[
    G = O
    \begin{bmatrix}
      \sqrt{d_1} & & & \\
      & \sqrt{d_2} & & \\
      & & \ddots & \\
      & & & \sqrt{d_{g}}
    \end{bmatrix}.
\]
Then $A = GG^t$ and $G \in \Sl_g(\R)$.
\end{proof}

\begin{proposition}\label{prop:A-over-detA}
  Let $S = {\{A/(\det A)^{\frac{1}{g}} | A \in \detl\}}$. Then $S$ is dense in $\permat$ with respect to the Euclidean topology.
\end{proposition}

\begin{proof}
%Let
%  \[
%   S =  \{A/(\det A)^{\frac{1}{g}} | A \in \detl\}.
%  \]
  (I don't understand the next sentence. Can it be made more precise?) By Lemma~\ref{lemma:ggt-spd-detl}, in the definition of $S$ we may replace $A$ with $\ell^n GG^t$ with $G \in \Sl_g(Z[1/\ell])$, $n$ sufficiently large (depending on $G$). Since $\det (\ell^n GG^t) = \ell^n$, we have
  \[
    S = \{GG^t | G \in \Sl_g(\Z[1/\ell])\}.
  \]
  The proposition now follows from Lemmas~\ref{lemma:sl-z-1overl-dense-sl-r} and \ref{lemma:ggt-periodmatrices}.
\end{proof}

\begin{definition}
If $X$ is a subset of a complex manifold $M$, we say that $X$ is {\em holomorphically dense} in $M$ if every holomorphic function that vanishes on $X$ also vanishes on $M$.
\end{definition}

\begin{remark}
Note that if $X$ is dense in $M$ with respect to the Euclidean topology then $X$ is holomorphically dense in $M$, and if $X$ is holomorphically dense in $M$ then $X$ is Zariski dense in $M$.
\end{remark}

Let $\R^+$ denote the set of positive real numbers.
\begin{lemma}\label{lemma:holomorphic-closure-irM}
  The set ${\{ irM | M \in \permat, r \in \R^+\}}$ is holomorphically dense in $\hh_g$.
\end{lemma}


\begin{proof}
  The set $H := \{ irM | M \in \permat, r \in \R^+\}$ is the subset of $\hh_g$ with real part $0$. Note that $H$ is a real manifold with $\dim_{\R} H = g(g+1)/2$. So by \cite[Ch.~4.8]{krantz2017harmonic}, it is enough to show that $H$ is a totally real. That is, we need to show that the tangent space \textcolor{red}{TODO: figure out correct property}

  As $H$ is a topological group (it is isomorphic to $\R^+ \times \detl$), it is enough to consider the tangent space at the identity $iI_g$. The exponential map goes from \textcolor{red}{TODO: finish}
\end{proof}

% \begin{lemma}
%   The holomorphic closure of ${\bigcup_{A \in \detl} \psi_A(i\R^+)}$ is $\hh_g$.
% \end{lemma}

% \begin{proof}
% \end{proof}

% \begin{lemma}
%   \[
%     \hclos{\bigcup_{A \in \detl} \im \psi_A} = \hh_g.
%   \]
% \end{lemma}

% \begin{proof}
%   Since $\psi_A(i\R^+) \subset \im \psi_A$, the claim follows from the previous lemma.
% \end{proof}


\begin{proof}[Proof of Theorem~\ref{thm:invariant-c-constant}]
  We have
  \begin{align*}
    \cup_A \psi_A(i\R^+) &= \cup \{irA | r \in \R^+, A \in \detl\} \\
               &= \cup \{irA/(\det A)^{\frac{1}{g}} | r \in \R^+, A \in \detl\},
  \end{align*}
  and this set is dense in
  \[
    \{irM | r \in \R^+, M \in \permat\}
  \]
with respect to the Euclidean topology,  by Proposition~\ref{prop:A-over-detA}.
  By Lemma~\ref{lemma:holomorphic-closure-irM}, the set $\cup_A \psi_A(i\R^+)$ is holomorphically dense (define this) in $\hh_g$. Thus $\cup_A \im \psi_A$ is holomorphically dense in $\hh_g$. Passing to the quotient, if we consider $\ag$ as a complex manifold, we have that $\cup_A \im \psimod$ is holomorphically dense in $\ag$.

  By definition of $\psimod$,  $\im \psimod \subset \ag$ is the image of $\im \psi_A \subset \hh_g$ under the canonical covering map $\hh_g \to \ag$. As we have shown, $\im \psi_A$ is holomorphically dense in $\hh_g$, and so the image of $\psimod$ in $\ag$ is holomorphically dense when viewing $\ag$ as a complex manifold. But holomorphically dense implies Zariski dense, and the claim follows.
\end{proof}



\section{Proof of Theorem~\ref{thm:Sg-dense}}
\label{sec:step-2}

%The main goal of this section is to prove Theorem~\ref{thm:Sg-dense}.

\textcolor{red}{TODO: introduce lemmas}

\begin{lemma}\label{lem:silly}
  Let $\ell$ be a prime. Then there is an imaginary quadratic field $K$ such that $\sO_K^\times = \{\pm 1\}$ and $\ell$ splits into principal primes in $K$.
\end{lemma}
\begin{proof}
  The fields $K$ in which $\ell$ splits into principal primes are precisely the fields generated by roots of the polynomials $x^2 - tx + \ell$ with $0 < |t| < 2\sqrt{\ell}$. There are $2\lfloor 2\sqrt{\ell} \rfloor$ such polynomials. Recall that there are, up to Galois conjugates, at most $4$ elements of norm $\ell$ in $\Q(i)$ and at most $6$ elements of norm $\ell$  in $\Q(\sqrt{-3})$. Thus, if $4\sqrt{\ell} > 10$, then there are more than $10$ such polynomials, so one must correspond to a field other than $\Q(i)$ and $\Q(\sqrt{-3})$. The remaining primes $\ell < 7$ can be checked by hand.
\end{proof}

\begin{lemma}\label{lem:K-exists}
  Suppose $\ell$ is a prime, $g$ is a positive integer, and $K$ is an imaginary quadratic field over which $\ell$ splits into principal primes, namely $\ell = \alpha\sO_K \cdot \overline{\alpha}\sO_K$ with $\alpha\in \sO_K$.
%  \begin{enumerate}
%    \item $\sO_K^\times = \{\pm 1\}$,
%    \item $\ell$ splits in $K$ into principal primes $\alpha\sO_K \cdot \overline{\alpha}\sO_K$ with $\alpha\in \sO_K$,
%    \item
Then there are infinitely many rational primes $q$ such that
    \begin{enumerate}
      \item $q$ is inert in $K$,
      \item $q \equiv -1 \mod{g}$, and
      \item $\alpha \mod q\sO_K \in (\sO_K/q\sO_K)^\times$ is a $g$th power.
    \end{enumerate}
%  \end{enumerate}
\end{lemma}
\begin{proof}
  %By Lemma~\ref{lem:silly}, we can find a $K$ satisfying (1) and (2).
  Let $L = K(\zeta_g,\alpha^{1/g},\overline{\alpha}^{1/g})$.
  Let $\sigma$ be any complex conjugation in $\Gal(L/\Q)$.
  Let $\frak{q}$ be any prime of $L$ unramified in $L/\Q$ whose Frobenius $\mathrm{Frob}(\frak{q}) \in \Gal(L/\Q)$ is $\sigma$. The Chebotarev density theorem guarantees that there are infinitely many such $\frak{q}$.
  Let $q$ be the rational prime below $\frak{q}$.
  Since $\sigma|_K$ is complex conjugation, $q$ is inert in $K$, giving (1).
  Since $\sigma|_{\Q(\zeta_g)}$ is complex conjugation, $q \equiv -1 \mod{g}$, giving (2).
%To see this, note that under the usual isomorphism $\Gal(\Q(\zeta_g)/\Q) \cong (\Z/g\Z)^\times$, complex conjugation is associated to $-1$. The map Frobenius element corresponding to $\frak{q} \cap \Q(\zeta_g)$ sends $\zeta_g \mapsto \zeta_g^c$.
Since $\sigma$ has order $2$, we have $\sO_L/\frak{q} \cong \sO_K/q\sO_K \cong \F_{q^2}$. In particular, $\alpha^{1/g}$ is a $g$th root of $\alpha$ in $(\sO_K/q\sO_K)^\times$, giving (3).
%Moreover, if $\sigma$ satisfies these conditions, then so does any conjugate of $\sigma$.
%By Chebotarev's density theorem, it remains to show that there exists some $\sigma \in \Gal(L/\Q)$ of order $2$ that restricts to complex conjugation in $K$ and $\Q(\zeta_g)$. Every complex conjugation in $\Gal(L/\Q)$ has these properties.
%  Choose any embedding $\iota: L \to \C$ and define $\sigma \in \Gal(L/\Q)$ by $a \mapsto \iota^{-1}(\overline{\iota(a)})$. Note that $\sigma$ has order $2$. Because $K$ and $\Q(\zeta_g)$ are CM fields, $\sigma$ restricts to the usual complex conjugation on them. Therefore $\sigma$ has the desired properties.
\end{proof}

\textcolor{red}{TODO: introduce lemmas}

\begin{definition}
  Suppose that $K$ is an imaginary quadratic field. If $q$ is a prime number that is inert in $K$, let
  $$\Gt_q=(\sO_K/q\sO_K)^\times/(\Z/q\Z)^\times. $$
  %The field $K$ should be clear from the context.
\end{definition}

\begin{lemma}\label{lem:c-torsor}
  Suppose $E$ is an elliptic curve over an algebraically closed field $k$, and $\End(E) \cong \sO_K$ for some imaginary quadratic field $K$. Let $q$ be a prime inert in $K$ and not equal to the characteristic of $k$. Then the set of subgroups of $E$ of order $q$ is a $\Gt_q$ torsor.
\end{lemma}
\begin{proof}
  Note that $E[q]$ is a module over $\sO_K/q\sO_K \cong \F_{q^2}$  of size $q^2$. Therefore $E[q] \cong \F_{q^2}$ as $\F_{q^2}$-vector spaces. So $(\F_{q^2})^\times/(\F_q^\times) \cong \Gt_q$ acts freely and transitively on the one-dimensional $\F_q$-subspaces of $E[q]$.
\end{proof}

\begin{definition}[{\cite[Sec.~2]{kani2011products}}]\label{def:ker-idl}
  Let $A$ be an abelian variety over a field $K$. Given a regular left-ideal $I$ of $\End A$, let $H(I) = \cap_{\phi \in I}\ker \phi$. A finite subgroup scheme $H$ is a \emph{ideal subgroup} if $H = H(I)$ for some $I$. Symmetrically, given such an $H$, let $I(H) = \{\phi \in \End A \colon \phi(H) = 0\}$. We say $I$ is a \emph{kernel ideal} if it is the form $I(H)$ for some $H$.
\end{definition}

\begin{theorem}[{\cite[Thm.~20b]{kani2011products}}]\label{thm:kani-20b}
  Let $E$ be a CM elliptic curve over an algebraically closed field, and $H$ a finite subgroup scheme of $E$. Then $H$ is a kernel ideal if and only if there is an inclusion $\End E \hookrightarrow \End E/H$. Moreover, if $H_1,H_2$ are ideal subgroups of $E$, then $E/H_1 \approx E/H_2$ if and only if $I(H_1)$ is isomorphic to $I(H_2)$ as $\End(E)$ modules.
  %$f_{E_H} \mid f_E$
\end{theorem}

\begin{lemma}\label{lem:c-end}
  Suppose $E$ be an elliptic curve over an algebraically closed field, and $\End(E) \cong \sO_K$ for some imaginary quadratic field $K$. Let $q$ be a product of distinct primes not equal to the characteristic of the base field and inert in $K$. If $\sC$ is a subgroup of $E$ of order $q$, then $\End(E/\sC) \cong \Z + q\sO_K$. In particular, every endomorphism of $E/\sC$ is induced by an endomorphism of $E$ that takes $\sC$ to $\sC$.
\end{lemma}
\begin{proof}
  Factor $q = \prod_{i=1}^n q_i$, where each $q_i$ is a prime inert in $K$ and not equal to the characteristic of the base field. We can write $\sC = \sum_{i=1}^n \sC_i$ where each $\sC_i$ is a subgroup of order $q_i$.

  The isogeny $\pi: E \to E/\sC$ factors into a chain of isogenies
  \[
    E
    \xrightarrow{\pi_1}
    E/\sC_1
    \xrightarrow{\pi_2}
    E/(\sC_1 + \sC_2)
    \xrightarrow{\pi_3}
    \cdots
    \xrightarrow{\pi_n}
    E/\sum_{i=1}^n\sC_i.
  \]
  Let $E_j = E/\sum_{i=1}^j\sC_i$ so $\pi_j: E_{j-1} \to E_j$. Define $\sO_j$ to be the order $\End(E_j)$ in $K$, and let $f_j$ be the conductor of $\sO_j$. Since $\pi_j$ has degree $q_j$, we have $[\sO_j : \sO_{j-1}] = q_j$, $[\sO_{j-1} : \sO_j] = q_j$, or $\sO_{j-1} = \sO_j$ \cite[Prop.~5]{kohel1996endomorphism}. Therefore $[\sO_K : \sO_j]$ divides $\prod_{i=1}^{j}q_i$. Recall from \cite[Prop.~7.20]{cox2011primes} that ideals of $\sO_j$ that are prime to $f_j$ are in bijection with primes of $\sO_K$ avoiding $f_j$. In particular, there is no prime of $\sO_{j-1}$ of norm $q_j$. So $\ker\pi_j$ is not an ideal subgroup \cite[Prop.~23]{kani2011products}. Then by Theorem~\ref{thm:kani-20b}, $[\sO_{j-1} : \sO_j] = q_j$. It follows that $[\sO_K : \sO_n] = \prod_{i=1}^n q_i$ as required.
\end{proof}

\begin{lemma}\label{lem:c-subgps-distinct-quotients}
  Suppose $E$ be an elliptic curve over an algebraically closed field, and $\End E \cong \sO_K$ for some imaginary quadratic field $K$ with $\sO_K^\times = \{\pm 1\}$. Let $q$ be a prime not equal to the characteristic of the base field and inert in $K$. If $\alpha,\beta \in \sO_K$ are prime to $q$, then $E/\alpha(\sC) \approx E/\beta(\sC)$ if and only if $\alpha(\sC) = \beta(\sC)$.
\end{lemma}
\begin{proof}
  Note that $\alpha(\sC) = (\alpha\overline{\beta})(\beta(\sC))$ so up to replacing $\sC$ with $\beta(\sC)$ and $\alpha$ with $\alpha\overline{\beta}$, it is enough to prove the claim for $\beta = 1$.

  If $\alpha(\sC) = \sC$, then $E/\alpha(\sC) \approx E/\sC$, so it suffices to show the other direction. Let $\tilde{u}: E/\alpha(\sC) \to E/\sC$ be an isomorphism. We want to show that $\sC = \alpha(\sC)$.

  Let $\pi_{\sC}$ denote the quotient map $E \to E/\sC$. Note that $\alpha$ induces an isogeny $\tilde{\alpha}: E/\sC \to E/\alpha(\sC)$ defined by the equation $\tilde{\alpha}\circ\pi_{\sC} = \pi_{\alpha(\sC)} \circ \alpha$. The map $\tilde{u}\circ\tilde{\alpha}$ is an endomorphism of $E/\sC$. By Lemma~\ref{lem:c-end}, $\tilde{u}\circ\tilde{\alpha}$ is induced by some $\alpha' \in \sO_K$ that fixes $\sC$. That is, $\pi_{\sC}\circ\alpha' = \tilde{u}\circ\tilde{\alpha}\circ\pi_{\sC} = \tilde{u}\circ\pi_{\sC}\circ\alpha$. This shows that $\deg(\alpha) = \deg(\alpha')$\textcolor{red}{TODO: do we need $\alpha$ prime to be separable?}. Because $\alpha$ and $\alpha'$ are prime to $q$,
  \[
    \sC \times \ker(\alpha')
    \cong
    \ker(\pi_{\sC} \circ \alpha')
    =
    \ker(\tilde{u}\circ\pi_{\sC}\circ\alpha)
    =
    \ker(\pi_{\sC}\circ\alpha)
    \cong
    \sC \times \ker(\alpha).
  \]
  Therefore $\ker(\alpha) = \ker(\alpha')$, so $\alpha' = \alpha\circ u$ for some $u \in \sO_K^\times$. By hypothesis, $u = \pm 1$. So $\alpha(\sC) = \pm\alpha'(\sC) = \pm\sC = \sC$.

  %Alternate proof of existence of u: $E/\alpha(\sC)$ and $E/\sC$ have a unique ascending isogeny. As $\tilde{u}$ is an isomorphism, it must send the ascending kernel of $E/\alpha(\sC)$ to that of $E/\sC$. The ascending kernel is the image of $E[c]$ under the quotient. Therefore $\pi^{\vee}\circ\tilde{u}\circ\pi_{\alpha(\sC)}$ is divisible by $c$, and so factors through an automorphism of $E$.
\end{proof}

\begin{lemma}\label{lem:prod-equiv-torsor}
  Suppose $E$ be an elliptic curve over an algebraically closed field, and $\End E \cong \sO_K$ for some imaginary quadratic field $K$ with $\sO_K^\times = \{\pm 1\}$. Let $q$ be a prime not equal to the characteristic of the base field and inert in $K$. Let $\sC$ be a subgroup of $E$ of order $q$. If $\alpha_1,\dots,\alpha_g,\beta_1,\dots,\beta_g \in \sO_K$ are prime to $q$, then
  \[
    \prod_{i=1}^g E/\alpha_i(\sC) \approx \prod_{i=1}^g E/\beta_i(\sC)
    \,\Leftrightarrow\,
    \prod_{i=1}^g \alpha_i \equiv \prod_{i=1}^g \beta_i \text{ in } \Gt_q.
  \]
\end{lemma}
\begin{proof}
  Let $\alpha = \prod_{i=1}^g\alpha_i$ and $\beta = \prod_{i=1}^g\beta_i$. Let $M$ be the diagonal matrix with diagonal entries $\alpha_1^{-1},\dots,\alpha_{g-1}^{-1}, \alpha_g^{-1}\alpha$. Note that $M\mod{c\sO_K} \in \Sl_g(\sO_K/q\sO_K)$. By \cite[Cor.~5.2, Pg.~18]{ktheory1964bass}, we can find $M' \in \Sl_g(\sO_K)$ such that $M \equiv M' \mod{q\sO_K}$. The matrix $M'$ corresponds to an automorphism of $E^g$ sending $\prod_{i=1}^g \alpha_i(\sC)$ to $\sC^{g-1} \times \left(\prod_{i=1}^g \alpha_i\right)(\sC)$. A similar construction with $\beta$ shows that
  \[
    \prod_{i=1}^g E/\alpha_i(\sC) \approx \prod_{i=1}^g E/\beta_i(\sC)
    \,\Leftrightarrow\,
    \left(E/\sC\right)^{g-1} \times E/\alpha(\sC) \approx \left(E/\sC\right)^{g-1} \times E/\beta(\sC).
  \]

  Note that $\alpha$ induces an isogeny $\tilde{\alpha}: E/\sC \to E/\alpha(\sC)$. Let $\sO = \Z + q\sO_K$, which by Lemma~\ref{lem:c-end} is isomorphic to $\End(E/\sC)$, $\End(E/\alpha(\sC))$, and $\End(E/\beta(\sC))$. Thus $\ker\tilde{\alpha}$ is an ideal subgroup by Theorem~\ref{thm:kani-20b}. The same holds for $\beta$ as well.

  %I(\phi) = \{f \in \End E/\sC \colon \ker f \supseteq \ker \phi \}
  For an isogeny $\phi$ with domain $E/\sC$, let $I_\phi$ denote the kernel ideal of $\ker(\phi)$, i.e. $I_\phi = \{f \in \End(E/\sC) \colon f(\ker(\phi)) = 0\}$.
  By \cite[Thm.~46]{kani2011products},
  \begin{align*}
    \left(E/\sC\right)^{g-1} \times E/\alpha(\sC) \approx \left(E/\sC\right)^{g-1} \times E/\beta(\sC)
    \\
    \Leftrightarrow
    \left(\bigoplus_{i=1}^{g-1} \sO\right) \oplus I_{\tilde{\alpha}} \cong \left(\bigoplus_{i=1}^{g-1} \sO\right) \oplus I_{\tilde{\beta}},
  \end{align*}
  where the second isomorphism is as $\sO$-modules. By \cite[Thm.~48]{kani2011products} (see also \cite[Rem.~49b]{kani2011products}), the latter isomorphism is equivalent to $I_{\tilde{\alpha}} \cong I_{\tilde{\beta}}$.
  By Theorem~\ref{thm:kani-20b},
  \[
    I_{\tilde{\alpha}} \cong I_{\tilde{\beta}}
    \,\Leftrightarrow\,
    E/\alpha(\sC) \approx E/\beta(\sC).
  \]
  By Lemma~\ref{lem:c-subgps-distinct-quotients},
  \[
    E/\alpha(\sC) \approx E/\beta(\sC)
    \,\Leftrightarrow\,
    \alpha(\sC) = \beta(\sC).
  \]
  By Lemma~\ref{lem:c-torsor}, $\alpha(\sC) = \beta(\sC)$ if and only if $\alpha \equiv \beta$ in $\Gt_q$.
\end{proof}

\begin{lemma}\label{lem:lim-degree}
  Suppose $A,A' \in \detl$, $\det(A) = \ell^n$, and $\det(A') = \ell^m$.
  %M_{g \times g}(\Z)$ be positive definite symmetric matrices such that $\det(A) = \ell^n$ and $\det(A') = \ell^m$ for a prime $\ell$. Let $X_A$ and $X_{A'}$ denote the images of the maps $\psi_A: Y_0(\ell^n) \to \ag$ and $\psi_{A'}: Y_0(\ell^m) \to \ag$.
  Then there exists a sequence $x_i \in Y_0(\ell^n)$ such that
  \[
    \lim_{i \to \infty}\#\left\{ y \in Y_0(\ell^m) \colon \psi_A(x_i) \approx \psi_{A'}(y) \right\} = \infty.
  \]
\end{lemma}
\begin{proof}
  For any imaginary quadratic field $K$ as in Lemma~\ref{lem:silly}, there exist a principal prime $\alpha\sO_K$ lying over $\ell$ and rational primes $q_i$ as in Lemma~\ref{lem:K-exists}. We may assume $q_i \neq \ell$ for all $i$. Let $E$ be an elliptic curve over an algebraically closed field with endomorphism ring $\End(E) \cong \sO_K$.

  If $\sC$ is a cyclic subgroup of $E$ of order prime to $\ell$, then $\alpha$ induces a chain of isogenies
  \[
    E/\sC \to E/\alpha(\sC) \to \cdots \to E/\alpha^n(\sC).
  \]
  Let $\alpha_{\sC,n}: E/\sC \to \cdots \to E/\alpha^n(\sC)$ be the composition of this chain. Then $\alpha_{\sC,n}$ is a cyclic $\ell^n$-isogeny, so the pair $(E/\sC,\alpha_{\sC,n})$ defines a point in $Y_0(\ell^n)$.

    For each $i$, we fix a cyclic subgroup $\sC_i$ of $E$ of order $q_i$. Define $x_1$ to be $(E/\sC_1,\alpha_{\sC_1,n})$, $x_2$ to be $(E/(\sC_1 + \sC_2), \alpha_{\sC_1 + \sC_2,n})$, and similarly for $x_i$. Note that the $x_i$ are distinct points, as the curves all have different endomorphism rings by Lemma~\ref{lem:c-end}.

    Next we will construct the values of $y \in Y_0(\ell^m)$ such that $\psi_{A'}(y) = \psi_{A}(x_i)$. We first focus on the case $i = 1$. By Lemma~\ref{lem:psimod-weakly-isomorphic-to-product}, $\psi_{A}(x_1) \approx E/\alpha^{n_1}(\sC_1) \times \cdots \times E/\alpha^{n_g}(\sC_1)$, where $\ell^{n_1},\dots,\ell^{n_g}$ are the elementary divisors of $A$. By our choice of $K$, $\alpha$ is a $g$th power in $\sO_K/q_1\sO_K$, so we can find $\gamma \in \sO_K$ such that $\alpha \equiv \gamma^g \mod{q_1\sO_K}$. By Lemma~\ref{lem:c-torsor}, $\alpha(\sC_1) = \gamma^g(\sC_1)$ and
    \[
      \psi_{A}(x_1) \approx E/\gamma^{gn_1}(\sC_1) \times \cdots \times E/\gamma^{gn_g}(\sC_1).
    \]
    Let $\ell^{n_1'},\dots,\ell^{n_g'}$ be the invariant factors of $A'$, and let $S$ denote a set of representatives $\beta \in \sO_K$ of the solutions to the equation
    \[
      \left(\gamma^{n_1 + \cdots + n_g}\right)^g \equiv \beta^g\left(\gamma^{n_1' + \cdots + n_g'}\right)^g
      \text{ in } \Gt_{q_1}.
    \]
    Note that $\#S = g$ since $g$ divides $q_1 + 1$, the order of $\Gt_{q_1}$. For all $\beta \in S$, let $y_\beta = (E/\beta(\sC_1),\alpha_{\beta(\sC_1),m})$. By Lemma~\ref{lem:psimod-weakly-isomorphic-to-product}, $\psi_{A'}(y_\beta) = E/\gamma^{gn_1'}(\beta(\sC_1)) \times \cdots \times E/\gamma^{gn_g'}(\beta(\sC_1))$. Therefore $\psi_A(x_1) \approx \psi_{A'}(y_\beta)$ by Lemma~\ref{lem:prod-equiv-torsor}. Moreover, by Lemma~\ref{lem:c-subgps-distinct-quotients} the $y_\beta$ are distinct. Hence we have found $g$ points $y \in Y_0(\ell^m)$ with $\psi_{A}(x_1) \approx \psi_{A'}(y)$.

  For $x_2$, we use a similar construction. By the Chinese remainder theorem, the set of subgroups of $E$ of order $q_1q_2$ is a torsor over $\Gt_{q_1} \times \Gt_{q_2}$. Finding $\beta$ such that
  \[
    \psi_{A}(x_2) \approx \psi_{A'}\left(E/\beta(\sC_1+\sC_2),\alpha_{\beta(\sC_1+\sC_2),m}\right)
  \]
  reduces to finding solutions $\beta$ to the equation
  \[
    \left(\gamma^{n_1 + \cdots + n_g}\right)^g \equiv \beta^g\left(\gamma^{n_1' + \cdots + n_g'}\right)^g
    \text{ in } \Gt_{q_1} \times \Gt_{q_2}.
  \]
  Here $\gamma$ is chosen such that $\alpha \equiv \gamma^g \mod{q_1q_2\sO_K}$. There are precisely $g^2$ solutions $\beta$ to this equation. Continuing this construction for $i=3,4,\dots$, we find that for $x_i$ there are $g^i$ points $y \in Y_0(\ell^m)$ such that $\psi_{A}(x_i) \approx \psi_{A'}(y)$.
\end{proof}

\textcolor{red}{TODO: introduce proof}

\begin{proof}[Proof of Theorem~\ref{thm:Sg-dense}]
  Let $S$ denote the Zariski closure of $\sg \cap (X_A \times X_{A'})$ \textcolor{red}{TODO: should this be $\sg(F)$ for some alg. closed field $F$?}. By Lemma~\ref{lem:lim-degree}, $\sg \cap (X_A \times X_{A'})$ has an infinite number of geometric points. This implies that $\dim S \geq 1$. Suppose the dimension equals $1$, so that $S$ is a finite union of curves $V = \cup V_i$. The $V_i$ cannot all be horizontal components---that is, of the form $X \times \{z\}$---since this would contradict Lemma~\ref{lem:lim-degree}. Let $V'$ be $V$ with the horizontal components removed. Consider the projection $\pi_X: V' \to X$. Lemma~\ref{lem:lim-degree} implies that this map has unbounded degree. But $\pi_X$ on each irreducible component of $V'$ is nonconstant, and so $\pi_X|_{V'}$ has finite degree, yielding a contradiction. Therefore $\dim S \geq 2$, whence the claim follows.
\end{proof}







\section{Proof of Theorem \ref{thm:arbit-char}}
\label{sec:characteristic-p}

Let $A$ be a $g \times g$ diagonal matrix whose diagonal entries are positive integers, and having determinant $N$. Observe that the map $\psimod: Y_0(N) \to \ag$ factors through $\ao^g$, where we consider elements of $\ao^g$ to be products of $g$ elliptic curves with the product polarization. Write
\[
\psidiag: Y_0(N) \to \ao^g
\]
for the associated morphism. Thus, if the diagonal entries of $A$ are $d_1, \ldots, d_g$ and for $(E, C) \in Y_0(N)$, if we define $E_i$ to be $E/(N/d_i)C$, then
\[
  \psidiag(E,C) = E_1 \times \cdots \times E_g.
\]

\begin{theorem}\label{thm:char-p-version}
  The set $\bigcup_A \im\psidiag$ is Zariski dense in $\ao^g$.
\end{theorem}

\begin{proof}
 Let $E_0$ be an elliptic curve with CM by $\sO_K$ for some imaginary quadratic field $K$. Choose a prime $\ell$ that is inert in $K$ and fix a subgroup $\sC_0$ of $E_0$ of order $\ell$. Let $\pi_0$ denote the quotient map $E_0 \to E_0/\sC_0 =: E_1$. Note that by Kani (give ref), since $\sC_0$ is not a kernel ideal of $E_0$, it follows that $\End E_1 \cong \Z + \ell\sO_K$.

 For $i \ge 1$, let $\sC_i$ be any subgroup of $E_i$ of order $\ell$ other than $\pi_{i-1}(\sC_{i-1})$, and let $\pi_i$ be the quotient map $E_i \to E_i/\sC_i =: E_{i+1}$. Again, $\End E_{i+1} \cong \Z + \ell^{i+1}\sO_{i}$ by \textcolor{red}{TODO: cite Kani?}.

 For all $n \in \Z_{\ge 1}$, the composition $\pi^{(n)} = \pi_{n-1} \circ \cdots \circ \pi_0$ is a cyclic $\ell^n$-isogeny on $E$ because \textcolor{red}{TODO: there is no backtracking}.

 Let $S = \{E_i\}_{i=0}^\infty$. Let $(E_{n_1},\dots,E_{n_g}) \in S$ and let $A$ be the diagonal matrix with diagonal entries $\ell^{n_1},\dots,\ell^{n_g}$.  Choose any $n > \max\{n_i\}$, and let $\sC = \ker \pi^{(n)}$. Then $(E_0,\pi^{(n)}) \in Y_0(\ell^n)$ and
 \[
   \psidiag(E_0,\sC) = E_{n_1} \times \cdots \times E_{n_g}.
 \]
Thus, $S \subset \bigcup\psidiag(Y_0(\ell^n))$. Since $S$ is an infinite set of non-isomorphic elliptic curves, $S$ is dense in $\ao$. Therefore $S^g$ is dense in $\ao^g$.
\end{proof}










\newpage

\section{Other Step 1}

In this section we prove Theorem~\ref{thm:curves-dense}.


\subsection{June 27}

\begin{lemma}
  For any subset $S \subset \N^n$, there is a nonempty open cone of linear functionals which are uniquely minimized on $S$, with the minimum attained on the same element.
\end{lemma}
\begin{proof}
  Let $s = (s_1,\dots,s_n)$ be the lexicographically minimal element of $S$, and let $\lambda$ be any linear functional satisfying the strict inequalities
  \[
    \lambda_i > \sum_{j>i} s_j\lambda_j.
  \]
  In particular, $\lambda_i>0$ for all $i$. Such linear functionals clearly form a nonempty open cone in $\R^n$, so it will suffice to show that they are all uniquely minimized at $s$.

  For any other element $t \in S$, we have $t > s$ in lexicographic order. If $m$ is the first index where they differ, then
  \begin{align*}
    \sum_{1\le i\le n} \lambda_i t_i
    \ge
    \sum_{1\le i\le m} \lambda_i t_i
    \ge
    \sum_{1\le i\le m} \lambda_i s_i
    +
    \lambda_m
    \sum_{1\le i\le n} \lambda_i s_i.
  \end{align*}
\end{proof}
%Lemma. For any subset S\subset \N^n, there is a nonempty open cone of
%linear functionals which are uniquely minimized on S, with the minimum
%attained on the same element.
%
%Proof.  Let s be the lexicographically minimal element of S, and let
%\lambda be any linear functional satisfying the strict inequalities
%
%\lambda_i > \sum_{j>i} s_j\lambda_j.
%
%(In particular, \lambda_i>0 for all i.)  Such linear functionals clearly
%form a nonempty open cone in \R^n, so it will suffice to show that they
%are all uniquely minimized at s.
%
%For any other element t\in S, we have t>s in lexicographic order.  If m
%is the first index where they differ, then
%
%\sum_{1\le i\le n} \lambda_i t_i
%\ge
%\sum_{1\le i\le m} \lambda_i t_i
%\ge
%\sum_{1\le i\le m} \lambda_i s_i
%+
%\lambda_m
%
%\sum_{1\le i\le n} \lambda_i s_i.
%QED

\begin{corollary}
  For any set $S$ of positive semidefinite integral quadratic forms, there is a nonempty open cone of positive definite symmetric matrices $M$ such that $Q\mapsto \tr(M\cdot Q)$ has a unique minimum on $S$.
\end{corollary}
\begin{proof}
  Let $M_1,\dots,M_{d(d+1)/2}$ be a sequence of positive definite symmetric matrices with integer entries that form a basis over $\R$ of the space of all symmetric matrices. Then the set of positive semidefinite integral quadratic forms injects linearly in $\N^{d(d+1)/2}$ via
  \[
    Q\mapsto (\tr(Q M_1),\dots,\tr(Q M_{d(d+1)/2}).
  \]
  If $\lambda$ is any element of the cone coming from the Lemma, then $M  = \sum_i \lambda_i M_i$ is positive definite, and thus the claim follows.
\end{proof}
%Corollary.  For any set S of positive semidefinite integral quadratic
%forms, there is a nonempty open cone of positive definite symmetric
%matrices M such that Q\mapsto \Tr(M\cdot Q) has a unique minimum on S.
%
%Proof.  Let M_1,\dots,M_{d(d+1)/2} be a sequence of positive definite
%symmetric matrices with integer entries that form a basis over \R of the
%space of all symmetric matrices.  Then the set of positive semidefinite
%integral quadratic forms injects linearly in \N^{d(d+1)/2} via
%
%Q\mapsto (\Tr(Q M_1),\dots,\Tr(Q M_{d(d+1)/2}).
%
%If \lambda is any element of the cone coming from the Lemma, then
%M=\sum_i \lambda_i M_i is positive definite, and thus the claim follows.
%QED

\begin{corollary}
  For any set $S$ of positive semidefinite integral quadratic forms, there is an element $g \in \Sl_d(\Z[1/\ell])$ such that $Q \mapsto \tr(Q g g^t)$ has a unique minimum on $S$.
\end{corollary}
\begin{proof}
  Indeed, the open cone from the previous corollary intersects the dense set of positive definite symmetric matrices of the form $r g g^t$ with $r > 0$, $g \in \Sl_d(\Z[1/\ell])$, and thus since it is a cone intersects the subset with $r = 1$.
\end{proof}
%Corollary.  For any set S of positive semidefinite integral quadratic
%forms, there is an element g\in \SL_d(\Z[1/l]) such that Q\mapsto \Tr(Q
%g g^t) has a unique minimum on S.
%
%Proof.  Indeed, the open cone from the previous corollary intersects the
%dense set of positive definite symmetric matrices of the form r g g^t
%with r>0, g\in \SL_d(\Z[1/l]), and thus since it is a cone intersects
%the subset with r=1.
%QED

\subsection{June 28}

Let $A$ be a symmetric positive definite matrix with determinate $N = \ell^n$. Let $X_A$ denote the image $\psi_A(Y_0(\det A))$. Our goal is to show that $\bigcup X_A$ is dense in $\ag$.

Note that $\psi_A$ pulls back Siegel modular functions from $\ag$ to $X_0(N)$. On the level of Siegel upper half spaces, the pull back is given by
\[
  f \mapsto = \{\tau \mapsto f(A\tau)\}.
\]

Now $\bigcup X_A$ is dense in $\ag$ if and only if for all nonzero Siegel modular functions $f: \ag \to \C$, there is some $\psi_A$ such that $f \circ \psi_A \neq 0$.

\begin{definition}
  A $g \times g$ matrix $N$ is \emph{half-integral} if $2N_{i,j} \in \Z$ and $N_{i,i} \in \Z$ for all $i,j$.
\end{definition}

\textcolor{red}{TODO: this is prob unnecessary, just cite Klingen}

\begin{lemma}
  Let $f$ be a holomorphic function on $\hh_g/\Sp_{2g}(\Z)$. Then $f$ can be written as
  \[
    f(\tau) = \sum_{N \text{ h.i.}} c_N \exp(2 \pi i)^{\tr(N\tau)}.
  \]
\end{lemma}
\begin{proof}
  This is given more rigorously in \cite[Pg.~44]{klingen1990introductory}.

  By definition, $f$ is a function on $\hh_g$ such that $f(\tau) = f(\gamma(\tau))$ for any $\gamma \in \Sp_{2g}(\Z)$. Let $S$ be any integral symmetric $g \times g$ matrix, and let $\gamma = \begin{bmatrix} I & S \\ 0 & I \end{bmatrix}$. Then $f(\tau) = f(\gamma(\tau)) = f(\tau + S)$. If we write $\tau = (\tau_{i,j})$, then $f$ is periodic in each variable $\tau_{i,j}$ with $i \leq j$. Thus we write $f$ as a multivariable Fourier series,
  \begin{align*}
    f(\tau_{i,j})
    &=
    \sum_{N \in \mathrm{Sym}_g(\Z)} a_N \prod_{i \leq j} \exp(2 \pi i \tau_{i,j})^{N_{i,j}}
    \\
    &=
    \sum_{N} a_N \prod_{i,j} \exp(2 \pi i)^{\tau_{i,j} N_{i,j}}
  \end{align*}
  where in the last line $N$ ranges over all half-integral symmetric $g \times g$ matrices. A short computation shows that $\sum_{i,j}\tau_{i,j}N_{i,j} = \tr(\tau N)$.
  %\sum_{i,j} \tau_{i,j}N_{i,j} = \sum_{i} \sum_{k} \tau_{i,k}N_{k,i} = \tr(\tau N).
\end{proof}

\textcolor{red}{TODO: this is prob unnecessary, just cite}

\begin{theorem}[Koecher principle]
  Let $f$ be a holomorphic function on $\hh_g/\Sp_{2g}(\Z)$, with Fourier series $f(\tau) = \sum_N a_N \exp(2 \pi i)^{\tr(\tau N)}$, where $N$ runs over all symmetric half-integer matrices. If $a_N$ is not positive semi-definite, then $a_N = 0$ for any $N$ that is not positive semi-definite.
\end{theorem}
\begin{proof}
  See the proof of \cite[Thm.~1, Pg.~45]{klingen1990introductory} or \cite[Thm.~2, Pg.~191]{bruinier2008the123}.

  %See also https://pdfs.semanticscholar.org/b1f4/d21acdbb599fe177fb15e087dcda2b7c2905.pdf
\end{proof}

Let $f$ be a modular function on $\hh_g/\Sp_{2g}(\Z)$ and $\tau \in \hh_g$. Then
\[
  (f \circ \psi_A)(\tau) = \sum_Q c_Q \exp(2 \pi i \tau)^{\tr(AQ)}.
\]
where $Q$ runs over all symmetric positive semi-definite half-integral $g \times g$ matrices.
Let $q = \exp(2 \pi i \tau)$. Then we can rewrite the previous equation as
\[
  (f \circ \psi_A)(\tau) = \sum_Q c_Q q^{\tr(AQ)}
\]

\textcolor{red}{TODO: this is prob unnecessary, just cite}

\begin{lemma}\label{lem:finite-fixed-trace}
  For any fixed number $t$, there are finitely many $g \times g$ symmetric positive semi-definite half-integral $g \times g$ matrices $Q$ such that $\tr(Q) \leq t$.
\end{lemma}
\begin{proof}
  This is at the bottom of \cite[Pg.~46]{klingen1990introductory}.

  Recall that $\langle A,B \rangle \mapsto \tr(AB^t)$ is an inner product on $M_{g \times g}(\R)$ and $M_{g \times g}(\Z)$ is a discrete subset. So it is enough to show that $\tr(Q^2)$ is bounded. Let $\lambda_i$ denote the eigenvalues of $Q$, with multiplicity. By hypothesis, $\lambda_i \in \R$ and $\lambda_i \geq 0$ for all $i$. Then $\tr(Q^2) = \sum \lambda_i^2 \leq (\sum \lambda_i)^2 = \tr(Q)^2 \leq t^2$.
\end{proof}

\begin{lemma}\label{lem:bound-trace-finite}
  For any positive definite symmetric real $g \times g$ matrix $A$ and $t \in \R$, the set of symmetric positive semi-definite half-integral $g \times g$ matrices $Q$ such that $\tr(AQ) \leq t$ is finite.
\end{lemma}
\begin{proof}
  Recall that $\langle x,y \rangle = \tr(xy^t)$ is an inner product on the space of real matrices. Moreover, if $B$ is any positive semidefinite matrix, then $\tr(B^2) \leq \tr(B)^2$. So
  \begin{align*}
    \tr(BQ)^2 &\leq \tr(B^2)\tr(Q^2)
    &&\text{by Cauchy–Schwarz}
    \\
    &\leq \tr(B)^2\tr(Q)^2
    &&\text{By positive semidefinite.}
    %\tr(B^2) = \sum \lambda_i^2 \leq \left(\sum \lambda_i\right)^2 = \tr(B)^2.
  \end{align*}
  Therefore, if $\tr(AQ) \leq t$, then
  \[
    \tr(Q) = \tr(A^{-1}AQ) \leq \tr(A^{-1})\tr(AQ) \leq \tr(A^{-1})t.
  \]
  The claim then follows from Lemma~\ref{lem:finite-fixed-trace}.
\end{proof}

\textcolor{red}{TODO: common knowledge? See \url{https://math.stackexchange.com/questions/113842/is-the-product-of-symmetric-positive-semidefinite-matrices-positive-definite}}

\begin{lemma}\label{lem:non-neg-tr}
  Let $M$ and $N$ be symmetric positive semi-definite real $g \times g$ matrices. Then $\tr(MN) \geq 0$.
\end{lemma}
\begin{proof}
  As $M$ is positive semi-definite, it admits a symmetric square root, say $M^{1/2}$. Then $M^{1/2}NM^{1/2}$ is positive semi-definite because
  \[
    \vec{v}^t(M^{1/2}NM^{1/2})\vec{v}
    =
    (M^{1/2}v)^t N (M^{1/2}\vec{v})
    \geq 0.
  \]
  The last inequality follows from the fact that $N$ is positive semi-definite.
\end{proof}

\begin{lemma}\label{lem:unique-minimizer}
  Let $S$ be any nonempty subset of symmetric positive semi-definite half-integral $g \times g$ matrices $Q$. Then there exists an open set of positive definite real symmetric $g \times g$ matrix $A$ such that $\min_{Q \in S}\tr(AQ)$ is achieved by a unique $Q$.
\end{lemma}
\begin{proof}
  Define
  \begin{align*}
    S_{\leq t} &= \min\{ Q \in S \colon \tr(Q) \leq t \}
    \\
    t_0 &= \min\{ \tr(Q) \colon Q \in S \}
    \\
    t_1 &= \min\{ \tr(Q) \colon Q \in S \text{ and } \tr(Q) > t_0 \}.
  \end{align*}
  Note that for any $t$, $S_{\leq t}$ is finite by Lemma~\ref{lem:bound-trace-finite}. In particular, the minimums above are well defined and $t_1 > t_0$.

  Consider the set $\mathfrak{S}$ of symmetric matrices $E$ such that
  \begin{enumerate}
    \item $\tr(EQ_i) \neq \tr(EQ_j)$ for all $Q_i,Q_j \in S_{\leq t_0}$ with $Q_i \neq Q_j$.
    \item $E$ is positive definite.
    \item $I + E$ is positive definite.
  \end{enumerate}

  The first conditions is the complement of a finite union of hyperplanes. The second is an open set. Note that both conditions hold under scaling by a positive number. Since $I$ is positive definite, it contains an open neighborhood which is positive definite. For any $E$ satisfying the first two properties, we can scale $E$ small enough so that it will also satisfy the third. Therefore $\mathfrak{S}$ is non-empty and open.

  It remains to check that any $E \in \mathfrak{S}$ satisfies the conclusion. By Lemma~\ref{lem:non-neg-tr},
  \[
    \tr(Q(I + E)) = \tr(Q) + \tr(QE) \geq \tr(Q).
  \]
  So
  \[
    \inf\left\{ \tr(Q(I+E)) \colon Q \in S \right\}
    =
    \min\left\{ t_0 + \tr(QE) \colon Q \in S \text{ and } \tr(Q) = t_0 \right\}.
  \]
  By the construction of $\mathfrak{S}$, there is a unique $Q$ such that the last set is minimal.
\end{proof}

\begin{proof}[Proof of Theorem~\ref{thm:curves-dense}]
  \textcolor{red}{TODO: apply Lemma~\ref{lem:unique-minimizer} to the set of $Q$ such that $c_Q \neq 0$. Use openness to find a symmetric matrix $A \in \Gl(\Z[1/\ell])$ and then scale by a power of $\ell$.}
\end{proof}

\begin{comment}
From: "Rains, Eric M." <rains@caltech.edu>
Subject: Re: some observations
Date: Fri, June 28, 2019 at 9:54:26 AM PDT
To: Alice Silverberg <asilverb@uci.edu>

We're feeling a bit daunted by the positive characteristic case....

     Actually, now that I understand things a bit better, I think the
positive characteristic proof is actually a bit simpler than the current
characteristic 0 proof (as it doesn't require knowing anything about
holomorphic density in several variables).  Let me tell you the
characteristic 0 analogue of the argument...

     The basic idea in characteristic 0 is that the Zariski closure of
the images of X_0(l^n) is cut out by an ideal in the algebra of Siegel
modular forms.  Moreover, each A gives rise to a homomorphism from
Siegel modular forms to ordinary modular forms over \Gamma_0(l^n), given by

f\mapsto (\tau\mapsto f(A\tau)).

So to show that the union of images of \phi_A is dense, it suffices to
show that any nonzero Siegel modular form has nonzero image under one of
these homomorphisms.

     Since a Siegel modular form is 1-periodic in each coefficient, it
can be expressed as a series of the following form:

f(T) = \sum_Q c_Q \exp(2\pi\sqrt{-1}\Tr(QT))

where T ranges over the Siegel upper half-space and Q ranges over
positive semidefinite integral quadratic forms.  Specializing this to
A\tau gives the expression

f(A\tau) = \sum_Q c_Q q^{\Tr(AQ)},

where q=\exp(2\pi\sqrt{-1}\tau).  (Note that for any positive definite
real symmetric matrix A, |\{Q:\Tr(AQ)=t\}| is finite, essentially by
Cauchy-Schwartz, or more precisely by the inequality

\Tr(BQ)\le \Tr(AQ)\Tr(A^{-1}B)

for any positive semidefinite real symmetric matrix B (which follows
from Cauchy-Schwartz by writing both B and Q as sums of rank 1
matrices).  In particular, if \Tr(AQ)=t, then \Tr(Q)\le \Tr(A^{-1})t is
bounded, so there are only finitely many possibilities.)

     If A is chosen so that \Tr(AQ) attains its minimum on the support
of f at the unique point Q_0 (which can always be done, by the argument
of the previous E-mail), then we have

f(A\tau) = c_{Q_0} q^{\Tr(A Q_0)} + o(q^{\Tr(A Q_0)}),

and thus f(A\tau) is not identically 0.

     The only difference in finite characteristic is that we need to
rewrite the original Siegel modular form in terms of the variables
q_{ij}:=\exp(2\pi\sqrt{-1}T_{ij}) for 1\le i\le j\le d, at which point
it is, I believe, known that such expressions extend to finite
characteristic.  (That is, a Siegel modular form can be viewed as a
section of an appropriate line bundle on a compactification of {\cal
A}_g, that line bundle makes sense in finite characteristic, its
q-expansions around the appropriate cusp are as stated, and it interacts
with modular morphisms as expected.)

     Note that the argument of the previous E-mail that there's a
unique minimizing Q can be simplified somewhat.  Let A be any positive
definite real symmetric matrix.  Since |\{Q:\Tr(AQ)\le t\}| is finite
for all t, not only are there finitely many Q such that c_Q\ne 0 that
attain the minimum value t_0 of \Tr(AQ), there is a separation between
that value and the next smallest possible value t_1.  Let M be any
positive definite real symmetric matrix such that the minimum of \Tr(MQ)
subject to \Tr(AQ)=t_0 and c_Q\ne 0 is < t_1-t_0.  Then the minimum of
\Tr((A+M)Q) subject to c_Q\ne 0 is attained only when \Tr(AQ)=t_0, and
thus in particular for an open set of M, the minimum is uniquely attained.

     In fact, a sufficiently small perturbation in any direction
outside a finite set of hyperplanes will work: if M points in the
direction we want to perturb, then for \epsilon sufficiently small,
\epsilon A+\epsilon^2 M is positive definite and satisfies \Tr((\epsilon
A+\epsilon^2 M)Q)<t_1-t_0  when \Tr(AQ)=t_0 and c_Q\ne 0, so that
(1+\epsilon)A + \epsilon^2 M satisfies the desired condition, and thus
so does A + ((\epsilon^2)/(1+\epsilon))M.  In other words, the
minimizing Q is unique outside a locally finite complex of polyhedral cones.

                                             Eric
\end{comment}









\bibliographystyle{alpha}
\bibliography{./references}

\end{document}
