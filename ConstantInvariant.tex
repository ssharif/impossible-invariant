\documentclass{amsart}
\usepackage{amsmath, amsthm, amssymb,latexsym,enumerate,mathrsfs}
\usepackage[all]{xy}
\usepackage{enumitem}
\usepackage{hyperref}
\usepackage{invariant}

\title{Algebraic invariants constant on unpolarized isomorphism classes of abelian varieties}

%\author[E.\ Rains]{E.\ Rains}
%\address{}
%\email{}
%\author[?]{?}
%\address{}
%\email{}
%\author[A.\ Silverberg]{A.\ Silverberg}
%\address{Department of Mathematics, University of California, Irvine, CA 92697, USA}
%\email{asilverb@uci.edu}
%\subjclass[2010]{??}
%\keywords{abelian varieties}
%\thanks{Support for the research was provided by the Alfred P.~Sloan Foundation
%and the National Science Foundation.}

\begin{document}

\maketitle

\section{Introduction}
\label{sec:introduction}

Let $\ag$ be the moduli space of principally polarized abelian varieties. If $A, B$ are abelian varieties, we say $A$ and $B$ are \emph{weakly isomorphic}, written $A \approx B$, if they are isomorphic as unpolarized abelian varieties. Our goal is to show the following.
\begin{theorem}\label{thm:invariant-c-constant}
  Suppose $f: \ag \to X$ be a morphism of schemes over $\C$. If $f(A) = f(B)$ whenever $A \approx B$, then $f$ is constant.
\end{theorem}
We will also show that the theorem holds when $\C$ is replaced by either $\Q$ or $\Z$. We will deduce the theorem as a corollary of another result, stated below.

Let $R$ be a scheme, and for an $R$-scheme $S$ write $S[R]$ to be the set
\[
  \dlim_{T/R} S(T)
\]
where the union is over all $R$-schemes $T$. Let
\[
  \sg: \rschemes \to \mathcal{P}((\ag \times \ag)[R])
\]
be the functor from $R$-schemes to the power set of $(\ag \times \ag)[R]$ which, for $T$ an $R$-scheme, gives the set
\[
  \sg(T) = \{(A,B) \in (\ag \times \ag)(T) | A \approx B\}.
\]
Then we will show the following.
\begin{theorem}\label{thm:sg-c-dense}
  $\sg(\C)$ is Zariski dense in $\ag \times \ag$.
\end{theorem}

\paragraph{Overview.}

The idea is as follows. Choose a prime $\ell$. For $n$ an integer, we will construct maps $X_0(\ell^n) \to \ag$. For each such map, the image of a point $(E, \phi)$ will be an abelian variety isogenous to $E^g$. As $n$ and the choice of maps vary, we obtain a sequence of curves $X_i \subset \ag$. We will show that
\begin{enumerate}
    \item $\overline{\cup X_i} = \ag$ and
    \item for all $i,j$, $\overline{S_g \cap (X_i \times X_j)} = X_i \times X_j$.
\end{enumerate}
Here, the bar denotes Zariski closure. From these two properties, Theorem~\ref{thm:sg-c-dense} will follow. To obtain Theorem~\ref{thm:invariant-c-constant}, construct the fiber product
\[
  \xymatrix{
    \df \ar[r] \ar[d] & \ag \ar[d]^f \\
    \ag \ar[r]^f & X.
  }
\]
There is a natural map $\df \to \ag \times \ag$. The hypotheses on $f$ will imply that, viewed as functors, the latter map factors through $\sg$. Since the functor $\sg$ is ``dense'' in the sense of Theorem~\ref{thm:sg-c-dense}, we will have $\df = \ag \times \ag$, whence the claim follows.

\section{Matrix density results}
\label{sec:matr-dens-results}

\begin{lemma}
  
\end{lemma}
\end{document}
